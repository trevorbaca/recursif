\documentclass{article}
\usepackage[utf8]{inputenc}
\usepackage[papersize={11in, 17in}]{geometry}
\usepackage[absolute]{textpos}
\TPGrid[1in, 1in]{12}{24}
\usepackage{palatino}
\parindent=0pt
\parskip=24pt
\usepackage{setspace}
\setstretch{1.3}
\usepackage{nopageno}
\usepackage{eurosym}
\begin{document}

\begin{textblock}{12}(0, 0)
\textbf{POÈME RÉCURSIF} establishes a network of events that build up and tear down pieces of rhythmic structure at one and the same time. The piece's thousands of attack points structure hundreds of differently overlapping rhythmic cells; moment-to-moment changes in the piece's surface result from interplay of material in the horizontal and vertical directions at once. 

\textbf{Realization.} The number of players is unspecified. Parts may be assigned one to a player, more than one to a player or shared between players. Instrumentation is unspecified beyond the designation of sixty-four untuned parts as written into the score. A battery of African, Asian or Western percussion instruments will give the piece a certain character; found collections of wood, stone or metal will give the piece another character; the repurposing of objects in the concert hall, street or studio will give the piece a different character still. Choices of instrumentation, performance space and the distribution of parts should all be made so as to reinforce the shifting, dense character of the musical surface and the many attack points that make it up. Attacks are all to be precise, well-articulated and of exactly equal duration; the sustained parts of durations written into the score are meaningless and there should be no difference between, for example, a quarter note, on the one hand, and, on the other, a sixteenth note followed by three sixteenth rests. The piece can be played either uniformly quietly or uniformly loudly. In either case, changes in perceived dynamic are to be effected through the entrance and exit of groups of players as indicated in the score. Contrast dense sections of the piece with thinner sections and let the massiveness of the rhythmic patterning vary accordingly. Tempo, once chosen, must remain constant for any one performance but may vary from performance to performance as indicated.

\textbf{Duration.} 6 to 7 minutes.
\end{textblock}


\begin{textblock}{12}(0, 23)
\textit{The present version of \textbf{Po\`{e}me r\'{e}cursif} was finished in August 2003 and is appreciatively dedicated to Gy\"{o}rgy Ligeti and Beno\^{i}t Mandelbrot.}
\end{textblock}

\end{document}
